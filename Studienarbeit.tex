% !TeX spellcheck = de-DE
% !TeX encoding = utf8
% !TeX program = lualatex
% !TeX TXS-program:compile = txs:///lualatex/[--shell-escape]
% !BIB program = biber
% -*- coding:utf-8 mod:LaTeX -*-

% vv  scroll down to line 200 for content  vv


\let\ifdeutsch\iftrue
\let\ifenglisch\iffalse
\input{pre-documentclass}
\documentclass[
  fontsize=12pt, % Vorgabe von Herrn Müller
  a4paper,  % Standard format - only KOMAScript uses paper=a4 - https://tex.stackexchange.com/a/61044/9075
  oneside,  % oneside für einseitigen Druck, twoside für buchartigen druck
  bibliography=totoc,
  %               idxtotoc,   %Index ins Inhaltsverzeichnis
  %               liststotoc, %List of X ins Inhaltsverzeichnis, mit liststotocnumbered werden die Abbildungsverzeichnisse nummeriert
  headsepline,
  cleardoublepage=empty,
  parskip=half,
  %               draft    % um zu sehen, wo noch nachgebessert werden muss - wichtig, da Bindungskorrektur mit drin
  draft=false
]{scrbook}
\input{config}
\setcounter{tocdepth}{4}
%%%
% EN: Syntax highligthing using pygments package
\usepackage[chapter]{minted}
% EN: line numbers within page margins
% DE: Zeilennummern innerhalb vom Rand
\setminted{numbersep=5pt, xleftmargin=12pt, fontsize=\huge, baselinestretch=1}
%%%

%http://www.jevon.org/wiki/Eclipse_Pygments_Style
%\usemintedstyle{eclipse}
%
%\usemintedstyle{autumn}
%\usemintedstyle{rrt}
%\usemintedstyle{borland}
%\usemintedstyle{friendlygrayscale}
\usemintedstyle{friendly}

%EN: compatibility of packages minted and listings with respect to the numbering of "List." caption
%    source: https://tex.stackexchange.com/a/269510/9075
\AtBeginEnvironment{listing}{\setcounter{listing}{\value{lstlisting}}}
\AtEndEnvironment{listing}{\stepcounter{lstlisting}}
%EN: We use the Listing environment to have the nice bar. So, we also have to patch the "Listing" environment for consistent counters
\AtBeginEnvironment{Listing}{\setcounter{listing}{\value{lstlisting}}}
\AtEndEnvironment{Listing}{\stepcounter{lstlisting}}

% Abkürzungsverzeichnis
% Hier stehen alle Abkürzungen
\pagenumbering{arabic}
\newacronym{erp}{ERP}{Enterprise resource planning}
\newacronym{cli}{CLI}{Command-line interface}
\newacronym{os}{OS}{Operating system}
\newacronym{vps}{VPS}{Virtual Private Server}
\newacronym{vm}{VM}{virtuelle Maschine}
\newacronym{gui}{GUI}{Graphical User Interface}



\makeindex
\usepackage{setspace} % Vorgabe von Herrn Müller
\onehalfspacing
\begin{document}

%----------------------------------------------------------------------------------------
%	TITELSEITE
%----------------------------------------------------------------------------------------
% Im Anhang Credit (CC BY 4.0!) für https://www.overleaf.com/latex/templates/uppsala-university-template/jvjprsfnzgbj#.WzX_c9IzaUk
\begin{titlepage}

\newcommand{\HRule}{\rule{\linewidth}{0.5mm}} % Defines a new command for the horizontal lines, change thickness here

\center % Center everything on the page
 
%----------------------------------------------------------------------------------------
%	HEADING SECTIONS
%----------------------------------------------------------------------------------------

\textsc{\LARGE Studienarbeit}\\[1.5cm] % Name of your university/college
\includegraphics[scale=.15]{logos/Logo_Hochschule_Kempten.png}\\[1cm] % Include a department/university logo - this will require the graphicx package
\textsc{\Large Open Source ERP-Systeme}\\[0.5cm] % Major heading such as course name
%\textsc{\large Course code}\\[0.5cm] % Minor heading such as course title

%----------------------------------------------------------------------------------------
%	TITLE SECTION
%----------------------------------------------------------------------------------------

\HRule \\[0.4cm]
{ \huge \bfseries ERPNext \\ \Large - Installation und Geschäftsfall -}\\[0.4cm] % Title of your document
\HRule \\[1.5cm]
 
%----------------------------------------------------------------------------------------
%	AUTHOR SECTION
%----------------------------------------------------------------------------------------

\begin{minipage}{0.4\textwidth}

\emph{Autor:}\\
Lukas \textsc{Willburger}\\ % Your name
Matrikelnr. 322445 \\
Wirtschaftsinformatik, 6. Semester \\ \newline 


\emph{Professor: }\\
Prof. Dr. Erich \textsc{Müller} \\ \newline  \newline

% Unterschriftenfeld; Vorgabe von Herrn Müller
%\hrulefill \\
%\centering Lukas \textsc{Willburger}
\end{minipage}\\[1cm]

% If you don't want a supervisor, uncomment the two lines below and remove the section above
%\Large \emph{Author:}\\
%John \textsc{Smith}\\[3cm] % Your name

%----------------------------------------------------------------------------------------
%	DATE SECTION
%----------------------------------------------------------------------------------------




{\textcopyright\ 2018}\\[2cm] % Date, change the \today to a set date if you want to be precise

\vfill % Fill the rest of the page with whitespace

\end{titlepage}

\pagenumbering{gobble}
%tex4ht-Konvertierung verschönern
\iftex4ht
  % tell tex4ht to create picures also for formulas starting with '$'
  % WARNING: a tex4ht run now takes forever!
  \Configure{$}{\PicMath}{\EndPicMath}{}
  %$ % <- syntax highlighting fix for emacs
  \Css{body {text-align:justify;}}

  %conversion of .pdf to .png
  \Configure{graphics*}
  {pdf}
  {\Needs{"convert \csname Gin@base\endcsname.pdf
      \csname Gin@base\endcsname.png"}%
    \Picture[pict]{\csname Gin@base\endcsname.png}%
  }
\fi

%\VerbatimFootnotes %verbatim text in Fußnoten erlauben. Geht normalerweise nicht.
\input{commands}


%Eigener Seitenstil fuer die Kurzfassung und das Inhaltsverzeichnis
%\deftripstyle{preamble}{}{}{}{}{}{\pagemark}
%Doku zu deftripstyle: scrguide.pdf
%\pagestyle{preamble}
%\renewcommand*{\chapterpagestyle}{preamble}



%Kurzfassung / abstract
%auch im Stil vom Inhaltsverzeichnis
%\ifdeutsch
%  \section*{Kurzfassung}
%\else
%  \section*{Abstract}
%\fi
%
%... Short summary of the thesis ...
%
%\cleardoublepage


% BEGIN: Verzeichnisse

\iftex4ht
\else
  \microtypesetup{protrusion=false}
\fi

%%%
% Literaturverzeichnis ins TOC mit aufnehmen, aber nur wenn nichts anderes mehr hilft!
% \addcontentsline{toc}{chapter}{Literaturverzeichnis}
%
% oder zB
%\addcontentsline{toc}{section}{Abkürzungsverzeichnis}
%
%%%

%Produce table of contents
%
%In case you have trouble with headings reaching into the page numbers, enable the following three lines.
%Hint by http://golatex.de/inhaltsverzeichnis-schreibt-ueber-rand-t3106.html
%
%\makeatletter
%\renewcommand{\@pnumwidth}{2em}
%\makeatother
%

\tableofcontents

% Bei einem ungünstigen Seitenumbruch im Inhaltsverzeichnis, kann dieser mit
% \addtocontents{toc}{\protect\newpage}
% an der passenden Stelle im Fließtext erzwungen werden.

%Auflistung der Codezeilen
%\listoffigures

%Auflistung der Tabellen
%\listoftables

%Wird nur bei Verwendung von der lstlisting-Umgebung mit dem "caption"-Parameter benoetigt
%\lstlistoflistings
%ansonsten:
%\ifdeutsch
%  \listof{Listing}{Verzeichnis der Listings}
%\else
%  \listof{Listing}{List of Listings}
%\fi

%mittels \newfloat wurde die Algorithmus-Gleitumgebung definiert.
%Mit folgendem Befehl werden alle floats dieses Typs ausgegeben
%\ifdeutsch
%\listof{Algorithmus}{Verzeichnis der Algorithmen}
%\else
  %\listof{Algorithmus}{List of Algorithms}
%\fi
%\listofalgorithms %Ist nur für Algorithmen, die mittels \begin{algorithm} umschlossen werden, nötig
% Abkürzungsverzeichnis

\printnoidxglossaries
\thispagestyle{empty}
\iftex4ht
\else
  %Optischen Randausgleich und Grauwertkorrektur wieder aktivieren
  \microtypesetup{protrusion=true}
\fi

% END: Verzeichnisse


% Headline and footline
\renewcommand*{\chapterpagestyle}{scrplain}
\pagestyle{scrheadings}
\pagestyle{scrheadings}
\ihead[]{}
\chead[]{}
\ohead[]{\headmark}
\cfoot[]{}
\ofoot[\usekomafont{pagenumber}\thepage]{\usekomafont{pagenumber}\thepage}
\ifoot[]{}



%%%%%%%%%%%%%%%%%%%%%%%%%%%%%%%%%%%%%%%%%%%%%%%%%%%%%%%%%%%%%%%%%%%%%%%%%%%%%%
%
% Main content starts here
%
%%%%%%%%%%%%%%%%%%%%%%%%%%%%%%%%%%%%%%%%%%%%%%%%%%%%%%%%%%%%%%%%%%%%%%%%%%%%%%
\pagenumbering{arabic}
\clearpage
\setcounter{page}{1}
% Hier stehen gesammelt alle Inhaltskapitel drin
\spacing{1.5}
% Hier können die einzelnen Kapitel inkludiert werden. Sie müssen in den 
% entsprechenden .TEX-Dateien vorliegen. Die Dateinamen können natürlich 
% angepasst werden.
\chapter{Einleitung}
\label{chap:einl}
Die Auswahl und Einführung eines geeigneten \gls{erp}-Systems ist seit jeher normalerweise mit einem hohen Kostenfaktor und Aufwand verbunden. Doch es gibt nicht nur Big Player wie SAP, Microsoft, Oracle oder Infor auf dem Markt, sondern auch kleinere, unabhängige und vollkommen kostenlose Open Source Systeme. Dazu zählt unter anderem auch ERPNext, ein indisches \gls{erp}-System, das in dieser Arbeit behandelt werden soll. Hierbei sollen die verschiedenen Systemkomponenten und deren Installation veranschaulicht werden. Abschließend wird mithilfe von ERPNext ein ausführlicher Geschäftsvorfall durchgespielt. So zeigt sich, welche Vor- und Nachteile ein Open Source \gls{erp}-System wie ERPNext mit sich bringt und ob die Erwartungen an eine solche Software erfüllt sind. \\
Die zu dieser Arbeit dazugehörige Ausarbeitung von Klemens Stasius beschäftigt sich mit der Open Source Bewertung.
\chapter{Systembestandteile}
\label{chap:funkt}
Um die Installation mit ihren einzelnen Komponenten besser verstehen zu können, wird an dieser Stelle kurz auf die für ERPNext entwickelten Bestandteile eingegangen.

\section{Frappé Framework}
Frappé, die Firma hinter ERPNext hat eigens für das System ein eigenes Web-Framework entwickelt (\vgl \cite{FrappeIo}). Dieses in Python und JavaScript geschriebene Tool bildet die Grundlage für die Architektur von ERPNext. Es vereint sowohl Backend- als auch Frontend-Komponenten und liefert somit viele nützliche Funktionen, um nicht komplett \glqq from scratch\grqq\ starten zu müssen. Dazu zählt unter anderem eine REST-Schnittstelle, ein Admin-\gls{ui}, eine Benutzersteuerung, Unterstützung für E-Mail-Accounts sowie eine modulare Architektur.

\section{Frappé Bench}
Die Frappé Bench ist ein \gls{cli} um verschiedene Seiten und Apps zu verwalten (\vgl \cite{GhBench}). 
Eine App bezeichnet in diesem Kontext eine im Frappé Framework programmierte Anwendung, wie beispielsweise ERPNext. Diese kann dann auf einer Seite (= Webseite) installiert werden, die von der Bench verwaltet wird. So können auch mehrere Instanzen einer App auf verschiedenen Seiten deployed werden, was bei einem \gls{erp}-System durchaus sinnvoll ist. Durch die Verwaltungsfunktionen der Bench können ähnlich den Paketverwaltungstools von Linux-Betriebssystemen Apps jederzeit einheitlich installiert, deinstalliert und geupdatet werden.\\
Da die Bench als \gls{cli}-Anwendung kein \gls{gui} besitzt, wurde extra dafür eine Software namens Bench Manager geschrieben (\vgl \cite{GhBenchManager}). Diese basiert ebenso auf dem Frappé Framework und lässt in begrenztem Umfang die gleichen Funktionen wie das Kommandozeilen-Programm erledigen. \\
Darüber hinaus bietet diese Softwarekomponente komfortable Dienste, wie beispielsweise die Einrichtung eines SSL-Zertifikats für Seiten mit Let's Encrypt an (\vgl \cite{GhBenchSsl}).
\begin{figure}
  \centering
  \includegraphics[width=0.5\textwidth]{Bilder/Bench_vereinfacht.png}
  \caption{Bench (vereinfacht)}
  \label{fig:benchVereinfacht}
\end{figure}
\begin{figure}
  \centering
  \includegraphics[width=\textwidth]{Bilder/Bench.png}
  \caption{Architektur der Bench}
  \label{fig:bench}
\end{figure}
\chapter{Installation}
\label{chap:inst}

Für die Installation von ERPNext werden seitens Frappé diverse Möglichkeiten angeboten. Diese stehen alle inklusive Anleitung auf ihrem GitHub-Repository zur Verfügung (\vgl \cite{GhBench}).

Generell weist Frappé als lauffähige Systeme für den Betrieb von ERPNext alle Betriebssysteme außer Windows aus. Grund hierfür ist das \gls{cli} der Software, das unter Microsofts Betriebssystem nicht lauffähig zu sein scheint. Genauere Erläuterungen dafür werden vom Hersteller nicht geliefert. Als präferierte Systeme werden auf Linux-Seite Debian, Ubuntu und CentOS vorgeschlagen, die als besonders gut getestete Systeme gelten. Darüber hinaus werden noch Arch Linux und macOS genannt. \\
Von unserer Seite wurde für die manuelle Installation und das Installationsskript zu Testzwecken anfangs die Linux Distribution Xubuntu in der neuesten Version (18.04, \glqq Bionic Beaver\grqq) verwendet. Da sich diese Version jedoch als nicht geeignet herausstellte – dazu im nächsten Abschnitt mehr – wurde auf Xubuntu in einer älteren Version (16.04, \glqq Xenial Xerus\grqq) zurückgegriffen.
 
Für die 60-minütige Präsentation wurde ERPNext per Installationsskript auch auf einem \gls{vps} eingerichtet. Dieser verfügt über zwei virtuelle Kerne, 2 GB DDR3-Arbeitspeicher, 10 GB SSD-Festplattenspeicher und das Betriebssystem Debian (8.10, Jessie). Diese Konfiguration erwies sich schon als vollkommen ausreichend für unsere Zwecke.

\section{Installationsskript}
Die einfachste Form der Installation auf einem Produktivsystem stellt das \glqq Easy Install\grqq-Python-Skript dar. Dafür benötigt das System lediglich die Programmiersprache Python in Version 2.7, was bei allen gängigen Unix-Systemen der Fall sein dürfte. \\
Offiziell werden die Betriebssysteme Ubuntu 16.04, CentOS 7 und höher, sowie Debian 8 und höher für das Skript unterstützt. Bei einem genaueren Blick in die Python-Datei fällt aber auf, dass auch die Darwin-Plattform, genauer gesagt macOS in den Versionen 10.9 - 10.12 bedacht wurde.\\
Da ERPNext ursprünglich auf einem Ubuntu-Derivat in der Version 18.04 installiert werden sollte, schlug der Installer erwartungsgemäß fehl (\vgl Abbildung \ref{fig:fehlInst}).
\begin{figure}[H]
  \centering
  \includegraphics[width=\textwidth]{Bilder/Fehlgeschlagene_Installation.PNG}
  \caption{Fehlgeschlagene Installation aufgrund der OS-Version}
  \label{fig:fehlInst}
\end{figure}
Wieso hierbei nur ein zwei Jahre altes Betriebssystem\footnote{\Vgl \cite{Ubuntu}. Die Ursprungsversion von 16.04 wurde bereits am 21. April 2016 veröffentlicht. Da es sich jedoch um eine \gls{lts} Versionsnummer handelt, erhält die normale Ubuntu-Version Updates bis 2021, das Derivat Xubuntu bis 2019 (\vgl \cite{Xubuntu16}).}  unterstützt wird, erscheint fraglich.\\
Auch die Änderung einer Codezeile im Python-Skript brachte nicht den gewünschten Erfolg und das Programm wurde mit einer Fehlermeldung abgebrochen (\vgl Abbildung \ref{fig:fehlInst2}). Nach dem Zurückgreifen auf die unterstützte Version 16.04 verlief die Installation ohne Probleme.
\begin{figure}
  \centering
  \includegraphics[width=\textwidth]{Bilder/Fehlgeschlagene_Installation_2.PNG}
  \caption{Fehlgeschlagene Installation aufgrund eines fehlenden Ordners}
  \label{fig:fehlInst2}
\end{figure}
An dieser Stelle hat der Nutzer auch die Wahl zwischen einer Entwickler- oder einer Produktiv-Installation. Aus einem Forums-Diskussionsbeitrag geht hervor, dass der Unterschied darin liegt, dass lediglich im sogenannten\glqq developer mode\grqq\ neue Vorlagen für beispielsweise Reports erstellt werden können. Die Produktiv-Installation hingegen lässt lediglich Änderungen zu, die über das Frontend von ERPNext gesteuert werden können (\vgl \cite{ProdVsDev}). Darüber hinaus werden bei der Produktiv-Variante bereits diverse Management-Tools automatisch aktiviert. \\
Zudem können beide Installationen mittels Parameter von einem neuen User angelegt werden, sodass ERPNext/Frappé Bench nicht direkt vom Root-User installiert werden muss.
Somit reduziert sich die ganze Installation beispielsweise auf:
\begin{minted}[linenos=true,escapeinside=||]{bash}
wget https://raw.githubusercontent.com/frappe/bench/master/playbooks/install.py
sudo python install.py --production --user erpnextUser
\end{minted}
Dabei wird das Skript zuerst heruntergeladen und dann im Produktiv-Modus für den Benutzer \glqq erpnextUser\grqq\ gestartet. \\
Während der Ausführung des Skripts werden alle benötigten Abhängigkeiten heruntergeladen und konfiguriert. Dazu zählen unter anderem diverse Python-Tools, MariaDB für die Datenhaltung, Redis als Key-Value-Store und Nginx als Webserver. Zudem werden verschiedene Parameter abgefragt, wie zum Beispiel das Datenbank Root-Passwort oder das Administrator-Passwort für ERPNext. Hierbei fällt positiv auf, dass der Benutzer immer über den Stand der verschiedenen Installationsschritte auf dem Laufenden gehalten wird (\vgl Abbildung \ref{fig:aktInst}). Negativ – wenn auch sicherlich nützlich – ist eine angelegte Textdatei, die die zuvor angelegten Passwörter im Klartext enthält.
\begin{figure}[H]
  \centering
  \includegraphics[width=\textwidth]{Bilder/Aktueller_Stand_Installation.PNG}
  \caption{Aktueller Stand der Installation}
  \label{fig:aktInst}
\end{figure}

\section{Manuelle Installation}
Auch die manuelle Installation ist sehr übersichtlich gehalten. Bevor man das eigentliche \gls{erp}-System installiert, müssen jedoch noch einige vorausgesetzte Softwarebausteine heruntergeladen werden. Dazu gehören Git, Python-Tools, Pip, MariaDB, Ngnix, Nodejs, yarn, Redis, cron und wkhtmltopdf. Die ursprüngliche Anleitung setzt das Wissen zur Installation dieser Bestandteile voraus. Bei genauerem Hinsehen fällt aber auf, dass seitens Frappé auch an nicht besonders Linux-versierte Benutzer gedacht wurde. Es gibt extra nochmals eine Anleitung (\vgl \cite{GhGuide}), die bei den einzelnen Arbeitsschritten mehr ins Detail geht. Diese steht für Linux-Betriebssysteme (Debian, Ubuntu) und macOS zur Verfügung. \\
Bei der Installation von Pip, dem Paketmanager von Python, trat ein Problem auf, das zur Folge hatte, dass der Download des Programms nicht beendet werden konnte. Mit ein wenig Linux-Wissen war es aber kein Problem, dies zu beheben, da schon das Aufsplitten eines Kommandos in zwei Arbeitsschritte das gewünschte Ergebnis hervorbrachte. Bei der Installation wird man wie auch beim Skript nach den gewünschten Passwörtern für MariaDB und ERPNext gefragt.
Nach der Installation der vorausgesetzten Software geht die genauere ausführlichere Anleitung andere Wege, als die ursprüngliche Anleitung.
Die ausführlichere Version zeigt verschiedene Wege auf, Seiten und Apps mittels Bench anzulegen, wohingegen die ursprüngliche Anleitung sehr zielgerichtet erklärt, wie man ERPNext installiert. An dieser Stelle wäre es wünschenswert gewesen, einfach \emph{eine} genaue Anleitung gehabt zu haben, anstatt zwei verschiedene, die kombiniert werden müssen.

\section{Virtuelle Maschine}
Die Installation über die \glp{vm} ist denkbar einfach. Hierfür muss lediglich von der ERPNext-Website ein Emulations-Image heruntergeladen werden. Dafür stellt Frappé ein Virtualbox-Image in der Produktiv-Variante und eines in der Entwickler-Variante bereit, sowie ein Vagrant-Image\footnote{Vagrant ist ein Ruby-Tool zur Orchestrierung von virtuellen Maschinen, vgl.\ \cite{Vagrant}.} in der Entwickler-Variante.\\
Wir haben uns für die Produktiv-Installation mit Virtualbox entschieden.
Dafür benötigt man eine obligatorische Virtualbox-Installation, sowie genügend Arbeits- und Festplattenspeicher für die Virtualisierung des Systems. Die Downloadgröße der Images beträgt jeweils 1.5 GB.\\ Nach dem erfolgreichen Start der \gls{vm} gibt das System nach einer kurzen Abfrage von \texttt{lsb\_release -a} folgende Informationen über sich preis:
\begin{minted}[linenos=true,escapeinside=||]{bash}
No LSB modules are available.
Distributor ID: Ubuntu
Description:    Ubuntu 16.04.4 LTS
Release:        16.04
Codename:       xenial
\end{minted}
ERPNext gibt auf ihrer Seite zwar an, dass es sich bei der virtuellen Maschine um Ubuntu in Version 14.04 handeln würde, wie hier aber ersichtlich wird ein Ubuntu 16.04 eingesetzt. Dies ist zwar auch nicht mehr die neueste Variante, aber immerhin wird hier auch auf eine \gls{lts} Version zurückgegriffen.\\
Da solche Systeme sowieso nur für Evaluierungszwecke gedacht sind, ist das verschmerzbar.
Das Image kommt ohne ein \gls{gui} aus und ist direkt nach dem Start schon vorkonfiguriert und bereit. Die Anleitung auf der ERPNext-Website gibt an, dass der User direkt auf seinem Host-System im Browser die Adresse \texttt{http://localhost:8080} aufrufen kann. Dies funktioniert auch erfolgreich, allerdings kommt an dieser Stelle ein nicht unbeträchtliches Problem auf: Die Standard-Login-Daten der Anleitung sind fehlerhaft. Somit kann man sich nicht ohne Ausprobieren von Varianten mit verschiedener Groß- und Kleinschreibung anmelden. Hier hat im Vergleich das Installationsskript durch die direkten Abfragen der Passwörter während des Installationsprozesses seine Vorteile.

\section{Docker}
Als letzte Möglichkeit bietet Frappé eine Installation mittels Docker an. Diese befindet sich allerdings noch im Beta-Stadium. \\
Da sich Docker Technologien des Linux-Kernels zu Nutze macht, kann es einzelne Anwendungen in einem Container kapseln und verbraucht darüber hinaus weniger Ressourcen als eine herkömmliche \gls{vm}. 
Hierbei werden für gewöhnlich die verschiedenen Teilbereiche, wie beispielsweise Datenbanken, Webserver und Applikation in ihre eigenen Container gesetzt. Dadurch ist es einfacher, bestehende Komponenten im Gesamtsystem auszutauschen, ohne eine der anderen Komponenten unbrauchbar zu machen. 
Diesen Ansatz verfolgt auch das offizelle Release (\vgl \cite{GhFrappeDocker}), jedoch war dieses zum Zeitpunkt unseres Tests nicht lauffähig.

Dagegen gibt es ein lauffähiges Release von Seiten der Community (\vgl \cite{DockerERPNext}). Jenes scheint bei der Community durchaus ein gewisses Interesse zu wecken, wie die mehr als 10000 Pulls belegen. Dabei wird für jedes neue ERPNext-Release überprüft, ob noch alle Einstellungen funktionsfähig sind. Dies stellt bei dem von Frappé gebrauchten Rolling-Release-Modell einige Arbeit dar. Diese Version verwendet für die Funktionstüchtigkeit jedoch eine veränderte Version der Verwaltungsplattform Bench und legt auch Docker-ungebräuchlich alle Datenbanken und die Applikationslogik in einen einzelnen Container.
Für den Einsatz eines solchen Docker-Containers ist lediglich die Docker Community Edition erforderlich. Nach deren Installation kann mittels 
\begin{minted}[linenos=true,escapeinside=||]{bash}
docker pull lukptr/erpnext7
\end{minted}
das aktuellste Docker-File heruntergeladen werden.\\
Nun kann man vor dem ersten Start im Docker-File die beiden Umgebungsvariablen \texttt{MARIADB\_PASSWORD} und \texttt{ADMIN\_PASSWORD} nach Belieben anpassen.
Um den Container zu starten muss nur noch 
\begin{minted}[linenos=true, escapeinside=||]{bash}
docker run -d --name erpnext  -p 80:80 lukptr/erpnext7
\end{minted}
ausgeführt werden. \texttt{80:80} sorgt in diesem Fall für die Weiterleitung von Port 80 des Containers zu Port 80 des Host-Systems. Daraufhin kann im Browser erfolgreich die Adresse \texttt{http://localhost:80} aufgerufen und sich mit den zuvor festgelegten Daten angemeldet werden.
Interessant ist hierbei noch ein Feature seitens Docker, das den einfachen Zugriff auf das Dateisystem des Gast-Betriebssystems zulässt. Diesen Umstand kann man sich zu Nutze machen, wenn man eigene Erweiterungen für das \gls{erp}-System entwickeln will.

\section{Fazit}
Alle angebotenen Installationsmöglichkeiten gehen für einen Nutzer mit grundlegenden Linux-Kenntnissen leicht von der Hand. Dabei gilt es lediglich die unterstützten Betriebssysteme und Systemversionen zu beachten, damit keine unerwarteten Fehler auftreten. Sollten solche doch einmal erscheinen, sind diese in der Regel jedoch so aussagekräftig, dass der User damit problemlos eine Suchmaschine befragen kann. Dabei ist auch das Forum von ERPNext (\vgl \cite{ERPNextDiscuss}) sehr hilfreich, da dort innerhalb von Stunden auf Fragen geantwortet wird, nicht selten auch vom Gründer von Frappé, Rushabh Mehta, selbst.\\
Das Easy-Install-Skript ist wohl am einfachsten, um das System schnell in einer Produktiv- oder Entwicklerumgebung einzusetzen. Durch die interaktive Abfrage von Parametern ist die Installation sehr komfortabel. Da alle benötigten Komponenten schon mit installiert werden, muss sich der Nutzer nicht mehr um fehlende Bestandteile kümmern. Dass so auf eine einfache Einrichtung wert gelegt wurde, ist durchaus ein positiver Faktor im Vergleich zu anderen \gls{erp}-Systemen.\\
Da die manuelle Installation mehr Spielraum lässt, eignet sie sich perfekt für anspruchsvollere Anwender, die mehr Kontrolle über die Verteilung von Applikationskomponenten benötigen. Im Endeffekt liefert sie aber beim genauen Befolgen der offiziellen Anleitung das exakt gleiche Ergebnis wie das Installationsskript.\\
Für einen Test des Systems in der Lehre eignet sich die angebotene und veränderte Docker-Installation aus der Community besonders, da die sogenannten Docker-Images einfach verteilt und bei Bedarf wieder auf ihren Anfangszustand zurückgesetzt werden können. Wie im vorigen Abschnitt schon angesprochen, ist es zudem sehr einfach, Verzeichnisse zwischen Host und Gast auszutauschen, was eine Erweiterung des bestehenden Codes sehr einfach macht.
\chapter{Geschäftsvorfall}
\label{chap:vorfall}
Im Folgenden wird mit dem \gls{erp}-System ERPNext ein vorher festgelegter Geschäftsvorfall durchgeführt. \\
Um es so nah wie möglich an einem realen Unternehmen zu halten, wurde exemplarisch die fiktive \emph{Allgold Premium Molkerei GmbH} gewählt. Diese stellt im Allgäu bereits in dritter Generation hochwertige Molkereiprodukte her. Besonders bekannt ist sie für ihren wohlschmeckenden jungen Gouda.\\
Genau dieser soll auch im einem ausführlichen Geschäftsprozess hergestellt werden. Dafür haben wir uns für den gängigen Dreischritt \emph{Einkauf - Fertigung - Vertrieb} entschieden (\vgl Abbildung \ref{fig:geschVorfall}). Während der Durchführung werden auch an manchen Stellen Punkte angesprochen, die beim Experimentieren mit dem System besonders positiv oder negativ aufgefallen sind.
\begin{figure}[H]
  \centering
  \includegraphics[width=\textwidth]{Bilder/Geschaeftsvorfall.PNG}
  \caption{Übersicht Geschäftsvorfall}
  \label{fig:geschVorfall}
\end{figure}
\emph{Hinweis:} Da die Vielzahl an Bildern zum Prozess den Rahmen dieser Seminararbeit sprengen würde, sind diese im Anhang zu finden. An den jeweiligen Stellen sind selbstverständlich diesbezüglich Verweise zu finden.

\section{Einkauf}
Da bei den einzelnen Abschnitten des Geschäftsprozesses nicht jeder kleinste Teil gezeigt werden kann, wurden bereits Vorkehrungen getroffen. So ist der Lieferant und die Rohprodukte Milch und Lab, sowie das Fertigprodukt Gouda bereits angelegt und mit entsprechenden Einkaufspreisen und Parametern versehen. Somit kann auf die anderen Prozessschritte und die Auffälligkeiten tiefer eingegangen werden.

\subsection{Lieferantenauftrag anlegen}
Zu Beginn dieses Prozesses muss für die Herstellung des Goudas Milch eingekauft werden. Mit einem neuen Lieferanten, Bauer Anton, haben wir uns im Vorfeld auf eine Lieferung von 2000 Litern Milch geeinigt. Da dieser Geschäftspartner bereits im System angelegt ist, kann mit diesem Datensatz eine Verknüpfung hergestellt werden (\vgl Abbildung \ref{fig:liefAuftrag}). Weil der Artikel Milch schon vorher angelegt worden ist, und hierbei ein Standardeinkaufspreis von 50 Cent eingetragen wurde, wird uns hier für 2000\ell\ Milch ein Gesamtbetrag von 1000€ vorgeschlagen. Natürlich kann an dieser Stelle der Stückpreis individuell abgeändert und die Summe vom System ausgerechnet werden. \\
Klickt man auf das kleine Dreieck ganz rechts des Artikels wird dem Benutzer eine Art Detailansicht angezeigt (\vgl Abbildung \ref{fig:auftrDetail}). Dort kann er noch einige Einstellungen tätigen, die sonst von den Standardeinstellungen ersetzt werden würden. \\
Nützlich ist allgemein auch, dass man stets einfach zu verknüpften Einträgen springen kann. Dazu muss lediglich der kleine Pfeil in jenem Eingabefeld gedrückt werden, in dem man sich gerade befindet (\vgl Abbildung \ref{fig:verlArtikel}).\\
Sieht man sich den Screenshot einmal genauer an, erkennt man an manchen Stellen schlechte oder nicht vorhandene Übersetzungen, wie beispielsweise bei\ \glqq Get last purchase rate\grqq. Diese Vorkommnisse häufen sich und decken sich mit dem, was die offizielle Übersetzungsseite preis gibt\footnote{Erreichbar unter: \url{https://translate.erpnext.com/}.}. Nur etwas mehr als die Hälfte der Strings der deutschen Übersetzung wurden verifiziert. Der andere Teil ist entweder maschinell übersetzt, oder wird in der Software noch im englischen Original angezeigt. \\
Wird der Auftrag nun abgeschlossen, muss der Lieferant natürlich auch über den Auftrag informiert werden. Dazu wird automatisch ein Popup-Fenster eingeblendet, sobald auf buchen gedrückt wird. Dort kann eine Email versendet werden, inklusive des Auftrags in PDF-Form (\vgl Abbildung \ref{fig:auftrPdf}). Dies erfordert die Einrichtung einer geeigneten Mail-Adresse in den Einstellungen, über die dann die Benachrichtigungen verschickt werden können.

\subsection{Ware empfangen}
Nach dem Absenden eines Lieferantenauftrages und der Bearbeitung seitens des Lieferanten kommt natürlich auch der Zeitpunkt, zu dem die Ware im Unternehmen eintrifft. Um eine Aktion im Rahmen des getätigten Lieferantenauftrags zu unternehmen, wählt man diesen einfach aus. Dort bekommt man dann direkt die häufigsten Aktionen vorgeschlagen (\vgl Abbildung \ref{fig:auftrUebersicht}). Oben links wird durch Ampelfarben und Text, wie hier \glqq Um zu empfangen und abzurechnen\grqq\ stets der aktuelle Stand des Lieferantenauftrags angezeigt. Das sieht man auch weiter unten beim Artikel Milch, der gelb gefärbt und somit noch nicht im Lager gebucht ist. \\
Um nun den Erhalt der Ware zu bestätigen und auf das vorher festgelegte, oder ein anderes Lager zu buchen muss auf das Plus-Zeichen neben Kaufbeleg gedrückt werden. Dadurch werden die bereits vorhandenen Informationen direkt in die nächste Maske übernommen (\vgl Abbildung \ref{fig:kaufbeleg}). Ein einfacher Klick auf den Text Kaufbeleg würde lediglich alle offenen Kaufbelege zu diesem Lieferantenauftrag anzeigen. \\
Sollte die angenommene Menge kleiner sein als die Bestellmenge, also nur eine Teillieferung erfolgt sein, kann das hierüber umgesetzt werden. Auch hier gilt wieder: Ein Klick auf das kleine Dreieck bietet dem Benutzer mehr Einstellungsmöglichkeiten. Nach dem Speichern des Kaufbelegs werden nicht ausgefüllte Felder, oder nicht erledigte Tätigkeiten von der Software hervorgehoben. Da die Molkerei stets die frische ihrer Produkte und einwandfreie Rohstoffe garantiert, wurde beim Rohmaterial Milch ein Flag für eine Qualitätskontrolle bei Eingang gesetzt. Deshalb fordert das System auf, diese durchzuführen (\vgl Abbildung \ref{fig:mittQual}). An dieser Stelle wird ein neuer Benutzer der Systems wahrscheinlich erst einmal etwas irritiert sein, da nicht sofort ersichtlich ist, durch welche Schaltfläche diese Kontrolle angestoßen werden kann. Dafür muss nämlich wieder in die Detailansicht des betreffenden Artikels gewechselt werden. Dort existiert dann im Abschnitt Lager und Referenz ein dementsprechender Punkt. Das System schlägt dem Benutzer an geeigneter Stelle stets schon vorhandene Elemente vor (soweit existent), oder lässt direkt ein neues anlegen, ohne dass der User aus der aktuellen Maske springen muss (\vgl Abbildung \ref{fig:auswQual}). \\
Die Qualitätskontrolle selbst (\vgl Abbildung \ref{fig:qualPruef}) ist schnell geschehen, da die Molkerei nur den Gehalt von Antibiotika in der Milch prüft. Danach lässt sich der Kaufbeleg ohne weitere Probleme buchen. \\
Geht man nun auf den Lieferantenauftrag zurück, fällt auf, dass sich der Status oben links auf \glqq Abrechnen\grqq\ geändert hat. Über das Plus bei Eingangsrechnung kann nun auch diese hinzugefügt werden. Im Lieferantenauftrag ist der Status danach auf \glqq Abgeschlossen\grqq\ gewechselt, aber die Lieferung ist noch unbezahlt. Der User könnte nun eine Zahlung über den Lieferantenauftrag anlegen, jedoch würde dabei der Bezug zur Rechnung verloren gehen. Geht man nun über die Verlinkung in die Eingangsrechnung des Auftrags und führt diese durch, wird ebendiese Verknüpfung vom System selbstständig hergestellt (\vgl Abbildung \ref{fig:zahlung}). Danach ändert sich der Status auf grün und \glqq Bezahlt\grqq. So bleibt die Zahlung auch später besser zuordenbar und Rechnungen bleiben nicht unabsichtlich unbezahlt.

\section{Fertigung}
Auch beim Bereich Fertigung wurden im Vorfeld bereits einige Dinge für die Stückliste angelegt. Dazu zählt der Arbeitsgang Lab hinzufügen (\vgl Abbildung \ref{fig:arbGang}) mit dem Arbeitsplatz Kessel (\vgl Abbildung \ref{fig:arbPlatz}).

\subsection{Stückliste anlegen}
Um den Gouda der \emph{Allgold Premium Molkerei GmbH} herzustellen wird eine Stückliste benötigt (\vgl Abbildung \ref{fig:stListe}). Diese kann neben benötigter Materialien auch Arbeitsgänge beinhalten. In diesem Beispiel verwenden wir zehn Liter Milch, sowie eine Einheit Lab. Dazu kommt ein Arbeitsgang, bei dem das Lab zu der Milch in den Kessel geschüttet wird. Aufgrund mangelnder Expertise unsererseits bei der Käseherstellung soll dies für den Moment schon genügen. \\
Bei dieser Stückliste wird automatisch ein Häkchen für den Standard gesetzt. Dies wird auch so belassen, damit die neue Stückliste später automatisch ausgewählt wird. 
Etwas unübersichtlich ist hierbei, dass die \glqq Operation Time\grqq\ für den Arbeitsgang in Minuten und nicht in Stunden angegeben werden muss. Dies sieht man erst, wenn man in die Detailansicht eines Arbeitsganges wechselt. Verwechselt man das bei unabsichtlich, werden die Betriebskosten vollkommen verfälscht. \\
ERPNext bietet zudem noch eine Möglichkeit, um eventuell anfallenden Ausschuss beziehungsweise Abfall bei der Herstellung zu berücksichtigen. Da dies aber bei diesem Beispiel nicht der Fall sein wird, kann das getrost übersprungen werden.

\subsection{Fertigungsauftrag erstellen}
Anhand der Stückliste kann nun ein Fertigungsauftrag für Gouda erstellt werden (\vgl Abbildung \ref{fig:fertAuftr}). Trägt man den zu fertigenden Artikel ein, wird automatisch die Standardstückliste zu diesem Artikel ausgewählt. Auch mehrstufige Stücklisten sind hierbei möglich, um diese Arbeit aber in einem überschaubaren Rahmen zu halten, wurde darauf verzichtet und auf eine normale Baukastenstückliste zurückgegriffen. \\
Beim Eintrag der herzustellenden Menge werden anhand der gewählten Stückliste die erforderliche Mengenanzahl der einzelnen Artikel, sowie die benötigte Zeit für die Arbeitsgänge ermittelt. Daraus ergeben sich dann für den Arbeitsgang auch die geplanten Betriebskosten.
An und für sich sind Pflichtfelder stets durch einen gelben Hintergrund gekennzeichnet. Beim Speichern und Buchen fällt dann jedoch auf, dass Hier auch Lager eingetragen werden müssen, welche diese Kennzeichnung davor nicht getragen haben. Das Eingangslager ist für die fertigen Erzeugnisse notwendig und das Fertigungslager dementsprechend für die Lagerung während der Fertigung. Danach lässt sich der Fertigungsautrag ohne Probleme ins System einbuchen un dessen Status wechselt auf \glqq Nicht begonnen\grqq\ (\vgl Abbildung \ref{fig:fertNichtBeg}). Im gleichen Augenblick wird ein dazugehöriges Timesheet erstellt.

\subsection{Zeiterfassung pflegen}
Nun müssen die Produktion gestartet und die entsprechenden Materialien aus dem Lager transferiert werden (\vgl Abbildung \ref{fig:matTransfer}). Dabei könnte auch nur eine Teilmenge hergestellt werden, was wir in diesem Beispiel jedoch nicht tun werden. Alle benötigten \glqq Einzelteile\grqq\ sind verfügbar und können somit verschoben werden (\vgl Abbildung \ref{fig:lagBuchung}). Nach Beendigung des Fertigungsauftrages mit dem von uns erdachten Arbeitsschritts kann die benötigte Zeit ins System eingebucht werden (\vgl Abbildung \ref{fig:timeSheet}). Dabei wird zuerst die veranschlagte Zeit voreingetragen. Sollte der Mitarbeiter mehr oder weniger Zeit benötigt haben, könnte er das an dieser Stelle eintragen. Das hätte dann wiederum Einfluss auf die Kosten des Arbeitsganges und somit des Herstellungspreises unseres Goudas. In diesem Beispiel werden alle Tätigkeiten mit einem User durchgeführt. In der Realität wäre das natürlich nicht der Fall und es könnten auch Auswertungen anhand der gebuchten Zeiten erstellt werden. \\
Ist unser Gouda gefertigt, muss eine neue Lagerbuchung angestoßen werden, um die Roh-Artikel aus dem Lager zu nehmen und den Gouda ins Lager der Fertigerzeugnisse zu legen (\vgl Abbildung \ref{fig:beenFertig}).

\section{Vertrieb}
Für den Vertrieb des Goudas  hat ERPNext nicht nur die Möglichkeit Kunden über das \gls{erp}-System durch einen Mitarbeiter der Molkerei anzulegen, sondern auch den Bestellprozess direkt durch den Kunden über einen Webshop anzustoßen. Dadurch wird das System direkt zur Verkaufsplattform.\\
In diesem Beispiel will ein der Kunde \glqq Großhandel Gummersbach\grqq\ Gouda für sein Geschäft einkaufen.

\subsection{Kunden anlegen}
Der Kunde kann sich unter dem Reiter \glqq Anmelden\grqq\ der Webseite einen neuen Account anlegen. Dazu muss er seinen vollständigen Namen (bzw. den Namen des Geschäfts), sowie eine Email-Adresse angeben (\vgl Abbildung \ref{fig:regKunde}). Dadurch kommt er nie mit der Verwaltungsplattform, wie sie ein Mitarbeiter benutzt, in Berührung. Nach erfolgreicher Registrierung wird dem Kunden eine Email zugeschickt (\vgl Abbildung \ref{fig:regMail}). Nach Setzen eines Passworts kann auch schon im Webshop eingekauft werden. \\ 
Zu diesem Zeitpunkt ist bereits ein neuer Kundeneintrag in ERPNext generiert worden und für Mitarbeiter sichtbar.

\subsection{Verkaufsauftrag erstellen}
In herkömmlichen Webshops können gleich mehrere Einheiten desselben Produktes auf einmal ausgewählt werden. Das ist mit dem Webshop von ERPNext nicht möglich (\vgl Abbildung \ref{fig:webGouda}). Erst im Einkaufswagen können mehrere Einheiten ausgewählt und ausgecheckt werden (\vgl Abbildung \ref{fig:webChart}). Vor dem Absenden der Bestellung wird noch eine Rechnungs- und Lieferadresse benötigt (\vgl Abbildung \ref{fig:webWarenkorb}). Nach dem Aufgeben der Bestellung hat der Kunde noch die Möglichkeit, den Auftrag zu drucken.

\subsection{Zahlung tätigen}
Nach dem Eingang des Kundenauftrags in das System hat nun die Rechnungsstellung und Bezahlung zu erfolgen. Die Rechnung ist schnell erstellt und kann wieder per Email an den Kunden inklusive persönlicher Nachricht übermittelt werden. \\
Danach kann die Zahlung erfolgen. Eigentlich wäre es sinnvoller, diese direkt beim Bestellvorgang abzuwickeln. Dafür fehlt lediglich ein Payment-Gateway, dass die Bezahlung per Kreditkarte o.ä. gegen eine Gebühr erlaubt. Das kann durch Zahlungseinstellungen im \gls{erp}-System relativ einfach erfolgen. Dazu stellt ERPNext schon vorgefertigte Schnittstellenanbindungen zu Stripe, PayPal und Razorpay zur Verfügung. \\
Wir wollen unsere Bestellung aber auf die gebräuchliche Art, per Überweisung, bezahlen. Dazu wird über den Kundenauftrag eine neue Zahlung erzeugt und mit den dementsprechenden Parametern gefüllt (\vgl Abbildung \ref{fig:zahKunde}). Dabei fällt auf, dass die urprüngliche Zahlung von 11,40€ auf 11€ abgerundet wurde. Das scheint eine spezielle, voreingestellte Einstellung zu sein, die jeweils immer auf den vollen Euro abrundet. Für andere Währungssysteme mag dies Sinn ergeben, jedoch ist das im Euro-Raum so nicht gebräuchlich. \\
Der Kunde kann stets den aktuellen Status seiner Bestellung über seinen Zugang zum Webshop einsehen. Bei erfolgter Bezahlung wechselt dieser zum Beispiel auf \glqq Bezahlt und nicht ausgeliefert\grqq. 

\subsection{Ware ausliefern}
Die Auslieferung der angeforderten Ware ist der letzte Teil dieses Geschäftsprozesses. Dafür muss der Benutzer beim Kundenauftrag einen Lieferschein erstellen. An dieser Stelle wurde eine kleine Anpassung programmiert (\vgl Abbildung \ref{fig:jsPopup}). Per JavaScript kann clientseitig Code in die Anwendung integriert werden. Im Normalfall ist dies eher für Formulare, beispielsweise bei der Berechnung von Werten gedacht, nichtsdestotrotz kann auch ein Popup mit hilfreichen Informationen angezeigt werden. \\
Genauso wie anfangs eine Qualitätsprüfung beim Eingang der Milch eingestellt wurde, muss nun die Qualität auch beim Ausgang des Endproduktes Gouda geprüft werden. Dazu muss diese Qualitätsprüfung auf dem bereits erklärten Weg hinzugefügt werden. Hierbei sollte auch ein Aufwands- und Ertragskonto für den zu verbuchenden Betrag eingetragen werden. \\
Der zuvor hergestellte Gouda hat zudem die Besonderheit, dass er für eine bessere Zurückverfolgbarkeit über eine Chargennummer verfügt. So müssen beispielsweise bei einer Rückrufaktion nur die betroffenen Chargen und nicht alle Produkte geprüft werden. \\
Mit abschließender Buchung des Lieferscheins ist der gesamte Prozess von Einkauf bis Vertrieb abgeschlossen.

\section{Fazit}
Durch das Erarbeiten dieses Geschäftsprozesses und das Experimentieren mit dem System lässt sich gut ein abschließendes Resümee ziehen. \\
Insgesamt ist ERPNext sehr gut und angenehm zu benutzen. Durch das den mobilen Plattformen entledigte Design findet man sich sehr schnell intuitiv zurecht. Das Design lässt sich von dem anfangs vorgegebenen Lila, passend zur Farbe des ERPNext-Logos, sehr leicht auf das firmeneigene Corporate Design anpassen. Es gibt allgemein viele Einstellungen, um den jeweiligen Bedürfnissen der Anwender gerecht zu werden. Dazu zählen auch automatisierte Workflows, die der ein oder andere User schon von Programmen wie Microsoft Sharepoint kennen könnte. Will man tiefergreifende Anpassungen machen, ist dies mit ein wenig Code auf Client- wie auf Serverseite möglich. Dazu existiert auch eine Entwickler-Dokumentation, sowie integrierte Hilfen im \gls{erp}-System. \\
Die Navigation zwischen den einzelnen Bereichen gestaltet sich durch die Verlinkungen mittels der Pfeilchen sehr einfach. Wo Verknüpfungen existieren, ist es immer möglich in andere Bereiche zu wechseln, ohne den Umweg über das Hauptmenü nehmen zu müssen. Dort wo in Übersichten, wie beispielsweise dem Lieferantenauftrag, verwandte Formulare befüllt werden müssen, eignet sich das Plus-Zeichen hervorragend um die wichtigsten Parameter zu übernehmen.

Trotz alledem gibt es selbstverständlich auch Negatives zu berichten, das während dem Testen aufgetreten ist. \\
Bei der Erstellung eines Accounts oder einer Rechnung wird abschließend vom System eine Email gesendet, soweit dies eingerichtet ist. Hier war beobachtbar, dass diese meist sofort an der Zieladresse angekommen sind, jedoch manchmal einige Minuten Zeit benötigt haben. Ob dies an der zum Versenden eingesetzten Gmail-Adresse liegt, oder ob dieses Problem tiefer im System besteht, ist unklar. In diesem Szenario haben nur wenige Personen auf das System zugegriffen. Deshalb ist fraglich, wie es sich auf die Performance auswirkt, sollten einmal mehrere hundert Personen Emails über ERPNext verschicken müssen. \\
So gut die Verknüpfungen an den meisten Stellen sind, so ist es umso gravierender, wenn diese fehlen. Auf unseren Geschäftsvorfall bezogen bedeutete das beispielsweise, dass es keine Verbindung von Produkt und Lieferant gab. Somit kann man z.B. auch das Produkt Lab bei einem Bauer bestellen, der eigentlich nur Milch liefert. \\
Die Benutzerführung ist zudem nicht immer Konsistent. Meist übernimmt das Plus-Zeichen alle benötigten Werte, manchmal muss man aber für genau diese Funktionalität den Button \glqq Make\grqq\ rechts oben betätigen. Den durchschnittlichen User könnte das sehr verwirren. \\
Wie schon in den vorangegangenen Abschnitten zum Geschäftsvorfall schon angesprochen sind die Übersetzungen ins Deutsche, sowie die lokalen Einstellungen nicht immer existent. Man mag meistens erahnen können, was bestimmte Floskeln bedeuten sollen, allerdings wäre eine weiter fortschreitende Übersetzung seitens der Community sehr wünschenswert. Wieso die lokalen Datumseinstellungen, sowie das lokal gebräuchliche Komma statt einem Dezimalpunkt nicht schon von vornherein eingestellt sind und seltsame Rundungsregeln angewandt werden ist sehr fraglich. Es sollte lieber keine deutsche Version existieren, solange diese nicht vollständig auf die lokalen Gegebenheiten abgestimmt ist. \\
Meistens lief das System stabil und zeigte nur geringe Latenzen beim Speichern und Buchen von Einträgen. Trotz allem gab es mehrere Systemausfälle während dem erneuten Durchgehen des Geschäftsprozesses für diese Arbeit (\vgl Abbildung \ref{fig:fehlMeldung}). Dabei stüzte ERPNext aus unbekannten Gründen ab und ließ sich auch nicht mehr durch einen Neustart der Bench wieder initialisieren. Lediglich ein Neustart des Servers behob das Problem. Dabei wurde auch in den diversen Log-Files des Systems kein Fehler gefunden, der diesen Absturz rechtfertigen würde. Nach einiger Zeit war ERPNext auch wieder vollständig funktionstüchtig und ähnliche Probleme traten nicht mehr auf.

Zusammenfassend lässt sich sagen: Dieses \gls{erp}-System vereint gute, neue Ansätze und eine angenehm zu benutzende Oberfläche. Allerdings werden erdachte Konzepte nicht durchgehend umgesetzt und die Ausrichtung auf den deutschen Markt lässt öfter einige Anpassungen und Übersetzungen missen.
%\chapter{Customizing}
\label{chap:cust}


\pagenumbering{Roman}
\appendix
%Literaturverzeichnis
%\printbibliography
\include{Quellen}

%\renewcommand{\appendixtocname}{Anhang}
%\renewcommand{\appendixname}{Anhang}
%\renewcommand{\appendixpagename}{Anhang}
%\input{latexhints-german}
%\input{latexhints-minted-german}
%Abbildungen, die nicht im Text auftreten
\chapter{Abbildungen}
\begin{figure}[H]
  \centering
  \includegraphics[width=\textwidth]{Bilder/Lieferantenauftrag.PNG}
  \caption{Anlegen eines neuen Lieferantenauftrages}
  \label{fig:liefAuftrag}
\end{figure}
\begin{figure}[H]
  \centering
  \includegraphics[width=\textwidth]{Bilder/Lieferantenauftrag_Produktdetail.PNG}
  \caption{Detailansicht eines Artikels eines Lieferantenauftrags}
  \label{fig:auftrDetail}
\end{figure}
\begin{figure}[H]
  \centering
  \includegraphics[width=\textwidth]{Bilder/Milch_Link.PNG}
  \caption{Verlinkung des Artikels Milch}
  \label{fig:verlArtikel}
\end{figure}
\begin{figure}[H]
  \centering
  \includegraphics[width=\textwidth]{Bilder/Uebersicht_Lieferantenauftrag.PNG}
  \caption{Übersicht Lieferantenauftrag}
  \label{fig:auftrUebersicht}
\end{figure}
\begin{figure}[H]
  \centering
  \includegraphics[width=\textwidth]{Bilder/Kaufbeleg.PNG}
  \caption{Kaufbeleg}
  \label{fig:kaufbeleg}
\end{figure}
\begin{figure}[H]
  \centering
  \includegraphics[width=\textwidth]{Bilder/Mitteilung_Qualitaetspruefung.PNG}
  \caption{Mitteilung für eine erforderliche Qualitätsprüfung}
  \label{fig:mittQual}
\end{figure}
\begin{figure}[H]
  \centering
  \includegraphics[width=\textwidth]{Bilder/Qualitaetskontrolle_auswaehlen.PNG}
  \caption{Auswahlmöglichkeiten bei der Qualitätsprüfung}
  \label{fig:auswQual}
\end{figure}
\begin{figure}[H]
  \centering
  \includegraphics[width=\textwidth]{Bilder/Qualitaetspruefung.PNG}
  \caption{Übersicht der Qualitätsprüfung}
  \label{fig:qualPruef}
\end{figure}
\begin{figure}[H]
  \centering
  \includegraphics[width=\textwidth]{Bilder/Zahlung.PNG}
  \caption{Zahlung mit Rechnungsverknüpfung}
  \label{fig:zahlung}
\end{figure}
\begin{figure}[H]
  \centering
  \includegraphics[width=\textwidth]{Bilder/Auftrag_Mail.PNG}
  \caption{Lieferantenauftrag als PDF}
  \label{fig:auftrPdf}
\end{figure}
\begin{figure}[H]
  \centering
  \includegraphics[width=\textwidth]{Bilder/Arbeitsplatz.PNG}
  \caption{Arbeitsplatz \glqq Kessel\grqq}
  \label{fig:arbPlatz}
\end{figure}
\begin{figure}[H]
  \centering
  \includegraphics[width=\textwidth]{Bilder/Arbeitsgang.PNG}
  \caption{Arbeitsgang \glqq Lab hinzufügen\grqq}
  \label{fig:arbGang}
\end{figure}
\begin{figure}[H]
  \centering
  \includegraphics[width=\textwidth]{Bilder/Stueckliste.PNG}
  \caption{Stückliste für Gouda}
  \label{fig:stListe}
\end{figure}
\begin{figure}[H]
  \centering
  \includegraphics[width=\textwidth]{Bilder/Fertigungsauftrag.PNG}
  \caption{Fertigungsauftrag für Gouda}
  \label{fig:fertAuftr}
\end{figure}
\begin{figure}[H]
  \centering
  \includegraphics[width=\textwidth]{Bilder/Fertigungsauftrag_nicht_begonnen.PNG}
  \caption{Fertigungsauftrag (Herstellung noch nicht begonnen)}
  \label{fig:fertNichtBeg}
\end{figure}
\begin{figure}[H]
  \centering
  \includegraphics[width=\textwidth]{Bilder/Materialtransfer.PNG}
  \caption{Materialtransfer für die Herstellung}
  \label{fig:matTransfer}
\end{figure}
\begin{figure}[H]
  \centering
  \includegraphics[width=\textwidth]{Bilder/Lagerbuchung.PNG}
  \caption{Lagerbuchung für die Herstellung}
  \label{fig:lagBuchung}
\end{figure}
\begin{figure}[H]
  \centering
  \includegraphics[width=\textwidth]{Bilder/Timesheet.PNG}
  \caption{Buchung der Zeiten}
  \label{fig:timeSheet}
\end{figure}
\begin{figure}[H]
  \centering
  \includegraphics[width=\textwidth]{Bilder/Lagerbuchung2.PNG}
  \caption{Beendigung des Fertigungsauftrages}
  \label{fig:beenFertig}
\end{figure}
\begin{figure}[H]
  \centering
  \includegraphics[width=\textwidth]{Bilder/Registrierung.PNG}
  \caption{Registrierung eines neuen Kunden}
  \label{fig:regKunde}
\end{figure}
\begin{figure}[H]
  \centering
  \includegraphics[width=\textwidth]{Bilder/Registrierung_Mail.PNG}
  \caption{Bestätigung der Registrierung per Email}
  \label{fig:regMail}
\end{figure}

 
% Selbständigkeitserklärung
\addchap{Eidesstattliche Erklärung}
Hiermit erkläre und versichere ich, Lukas Willburger, Matrikel-Nr.\ 322445, dass ich die vorgelegte Studienarbeit mit dem Thema
\begin{quote}
\textit{ERPNext} \textit{- Installation und Geschäftsfall -}
\end{quote}
selbständig verfasst und keine anderen als die angegebenen Quellen und Hilfsmittel benutzt habe, wobei ich alle wörtlichen und sinngemäßen Zitate als solche gekennzeichnet habe.  Dies gilt auch für Zeichnungen, Skizzen, bildliche Darstellungen sowie für Quellen aus dem Internet.
\\[6ex]

Dietmannsried, den \today


\rule[-0.2cm]{5cm}{0.5pt}

\textsc{Lukas Willburger} 
  
\pagestyle{empty}
\renewcommand*{\chapterpagestyle}{empty}
\end{document}
