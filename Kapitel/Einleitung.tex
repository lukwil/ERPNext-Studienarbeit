\chapter{Einleitung}
\label{chap:einl}
Die Auswahl und Einführung eines geeigneten \gls{erp}-Systems ist seit jeher normalerweise mit einem hohen Kostenfaktor und Aufwand verbunden. Doch es gibt nicht nur Big Player wie SAP, Microsoft, Oracle oder Infor auf dem Markt, sondern auch kleinere, unabhängige und vollkommen kostenlose Open Source Systeme. Dazu zählt unter anderem auch ERPNext, ein indisches \gls{erp}-System, das in dieser Arbeit behandelt werden soll. Hierbei werden die verschiedenen Systemkomponenten und deren Installation veranschaulicht. Abschließend wird mithilfe von ERPNext ein ausführlicher Geschäftsvorfall durchgespielt. So zeigt sich, welche Vor- und Nachteile ein Open Source \gls{erp}-System wie ERPNext mit sich bringt und ob die Erwartungen an eine solche Software erfüllt sind. \\
Die zu dieser Arbeit dazugehörige Ausarbeitung von Klemens \textsc{Stasius} beschäftigt sich mit der Open Source Bewertung.